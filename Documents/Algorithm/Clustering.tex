%%%%%%%%%%%%%%%%%%%%%%%%%%%%%%%%%%%%%%%%%
% Journal Article
% LaTeX Template
% Version 1.4 (15/5/16)
%
% This template has been downloaded from:
% http://www.LaTeXTemplates.com
%
% Original author:
% Frits Wenneker (http://www.howtotex.com) with extensive modifications by
% Vel (vel@LaTeXTemplates.com)
%
% License:
% CC BY-NC-SA 3.0 (http://creativecommons.org/licenses/by-nc-sa/3.0/)
%
%%%%%%%%%%%%%%%%%%%%%%%%%%%%%%%%%%%%%%%%%

%----------------------------------------------------------------------------------------
%	PACKAGES AND OTHER DOCUMENT CONFIGURATIONS
%----------------------------------------------------------------------------------------

\documentclass{minimal}

\usepackage{blindtext} % Package to generate dummy text throughout this template 

\usepackage{tikz}
\usetikzlibrary{arrows.meta}
\usepackage{adjustbox}

% define style for ``not visible nodes and edges''
\tikzset{bled/.style={black!40, line width=1pt, fill=white}}

\usepackage{graphicx}



\usepackage[english]{babel} % Language hyphenation and typographical rules




\usepackage{hyperref} % For hyperlinks in the PDF

\title{Understandable Bayesian Recommendation Engine}
%\author{Martin Molan \\
%	Jozef Stefan International Postgraduate School  \\
%	\and 
%	T \\
%	His Company / University \\
%	}

% Hint: \title{what ever}, \author{who care} and \date{when ever} could stand 
% before or after the \begin{document} command 
% BUT the \maketitle command MUST come AFTER the \begin{document} command! 





\begin{document}    

% Print the title
%\maketitle

%----------------------------------------------------------------------------------------
%	ARTICLE CONTENTS
%----------------------------------------------------------------------------------------
 

Preferences, learned by the model, determine the characteristic of each instance (student) in the population. These preferences can be arranged into a vector $p = (p_1, p_2, \dots, p_n)$. Students can thus be imagined as a point in an $n$ dimensional space. Since $n$ is significantly higher than $2$ such high dimensional space cannot be plotted. In order to plot this high dimensional space, we use a TSNE dimensional reduction (non linear projection) technique to project the points form the original $n$ dimensional space to $2$ dimensions. Important characteristic of TSNE projection technique is that it retains \emph{closeness} -- instances that are similar in original $n$ dimensional space (meaning similar students) will be plotted close to one another in $2$ dimensional space.


\end{document}
